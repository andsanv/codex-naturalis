\documentclass[12pt]{article}
\usepackage[utf8]{inputenc}
\usepackage[T1]{fontenc}
\usepackage[italian]{babel}

\title{Peer-Review 1: UML}
\author{<Nome1>, <Nome2>, <Nome3>\\Gruppo <numero del vostro gruppo>}

\begin{document}

\maketitle

Valutazione del diagramma UML delle classi del gruppo <numero dell’altro gruppo>.

\section{Lati positivi}

Indicare in questa sezione quali sono secondo voi i lati positivi dell’UML dell’altro gruppo. Se avete qualche difficoltà, provate a simulare il gioco a mano, immaginandovi quali sono le invocazioni di metodo che avvengono in certe situazioni che vi sembrano importanti (ad esempio, la fusione delle isole oppure il calcolo dell’influenza).

\section{Lati negativi}

Come nella sezione precedente, indicare quali sono secondo voi i lati negativi.

\section{Confronto tra le architetture}

Individuate i punti di forza dell’architettura dell’altro gruppo rispetto alla vostra, e quali sono le modifiche che potete fare alla vostra architettura per migliorarla.

\end{document}
