\documentclass{article}

\title{Peer Review \#1}
\author{Samuele Pischedda, Angelo Prete, Gabriele Raveggi, Andrea Sanvito\\GC11}
\date{3 Aprile 2024}

\begin{document}

\maketitle

Review dell'UML Class Diagram del gruppo GC01.

\section{Lati positivi}

\begin{itemize}
  \item Implementazione della classe \texttt{Corner} semplice ed efficace.
  \item Implementazione dello strategy pattern per calcolare i punti delle \\\texttt{ObjectiveCard}.
  \item Separazione di carte che possono essere giocate (\texttt{PlayableCard}) da quelle con scopo diverso (\texttt{ObjectiveCard}).
  \item Utilizzo delle enumerazioni per una migliore chiarezza e manutenibilità; interfaccia \texttt{PlayerResources} comune all'enum \texttt{Resource} e all'enum \texttt{Item}.
\end{itemize}

\section{Lati Negativi}
\subsection{Generali}

\begin{itemize}
  \item Tutti gli attributi nell'UML sono pubblici, nonostante siano stati aggiunti dei getter.
  \item Sintassi dell'UML a volte poco chiara (e.g.~non è esplicito se \texttt{Card} e \texttt{Deck} siano classi astratte o meno).
  \item Non sempre si è fedeli alla nomenclatura di gioco usata nel rulebook (e.g. ~RadixCards [StarterCards], Wolf [Animal], Mushroom [Fungi], etc).
  \item Mancanza di alcuni attributi e metodi essenziali al funzionamento delle classi (e.g. non si può risalire alle visible cards pescabili in \texttt{Room}, però la classe ha il metodo \texttt{getDrawableCards()}).
\end{itemize}

\subsection{Class-Specific}

\begin{itemize}
  \item La classe \texttt{GoldenCard}, sottoclasse di ResourceCard, non rispetta il principio di sostituzione di Liskov (dovrebbe invece essere sottoclasse di \\\texttt{PlayableCard}).
  \item Implementare il ``back'' di una carta \texttt{RadixCard} come sottoclasse di \texttt{RadixCard} non è molto chiaro.
  \item \texttt{RadixCard} è implementata come una carta giocabile, ma viene assegnata al giocatore a inizio partita.
  \item L'implementazione attraverso un Set di posizioni della board di gioco (\texttt{Field}) può risultare molto scomoda. Un'implementazione con una matrice o una \texttt{Map<Coordinate, PlayableCard>} potrebbe semplificare notevolmente la logica.
  \item L'interfaccia \texttt{CardResources} risulta ambigua a causa di valori sovrapposti, può contenere infatti sia una risorsa (o item) ma anche il valore \texttt{FULL}.
\end{itemize}

\section{Confronto tra le
  architetture}

\begin{itemize}
  \item L'implementazione della classe \texttt{Corner} è molto più semplice della nostra, che invece prevede delle sottoclassi in base al tipo di angolo (\texttt{VisibleCorner}, \texttt{HiddenCorner} e \texttt{CoveredCorner}).
  \item L'interfaccia comune per gli enum \texttt{Resource} e \texttt{Item} è interessante perché permette di eliminare ridondanza.
\end{itemize}

\end{document}
